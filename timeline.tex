\documentclass[tikz,border=10pt]{standalone}

\usepackage{fontspec}
\usepackage{luacode}

\setmainfont{HelveticaNeue-Light}

\begin{document}

\tikzset{annotation/.style   = {black!70!blue, very thin}}
\tikzset{hlannotation/.style = {black!50!red, very thick}}
\tikzset{bubble/.style      = {anchor=west,     annotation, sloped, at end, rounded corners=2pt, draw, scale=0.6, fill=gray!5 }}
\tikzset{hlbubble/.style    = {anchor=west,   hlannotation, sloped, at end, rounded corners=2pt, draw, scale=0.9, fill=red!20, text=black!50!red, semithick }}
\tikzset{titlebubble/.style = {anchor=center, hlannotation, sloped,         rounded corners=2pt, draw, scale=1.5, fill=red!10, text=black!50!red, semithick }}

\begin{tikzpicture}
\coordinate (A) at (0.15, 0.06);
\begin{luacode}

		years = {1995, 2005, 2010, 2015, 2020, 2025}
		s =     {   0, 0.05,  0.2, 0.6,  0.95,    1}
		x0 = 0
		x1 = 20


		-- helper functions --
		----------------------

		-- find index in years such that years[i] <= y < years[i+1]
		function findi(y)
			for i=1, #years-1 do
				if ( years[i] <= y ) and ( y < years[i+1] ) then
					return i
				end
			end
			return #years-1
		end


		-- compute X coordinate corresponding to (year, month) for drawing
		function dateX(year, month)
			y = year + ( month - 1 ) / 12
			i = findi( y )
			sy = s[i] + ( s[i+1] - s[i] ) * ( y - years[i] ) / ( years[i+1] - years[i] )
			xy = x0 + sy * ( x1 - x0 )
			return xy
		end



		-- read data from csv file --
		-----------------------------

		require('csv.lua')
		local data = dataToTable('ij timeline - Sheet1.csv')



		-- read data from "highlighted" file --
		---------------------------------------
		local file = io.open('highlighted')
			hlkey = file:read()
		file:close()
		local hlrow = 0
		for r=1, #data do
			if data[r][6] == hlkey then
				hlrow = r
				break
			end
		end


		-- create TikZ drawing --
		-------------------------


		-- presenter
		if ( hlrow > 0 ) then
			local presenter = data[hlrow][5]
			tex.sprint('\\node at (0,1) [anchor=north west, green!50!black, scale=0.7, inner sep=0, outer sep=0] {\\footnotesize(' ..presenter.. ')};')
		end


		-- draw timeline
		tex.sprint('\\draw [->] (' ..x0.. ',0) -- (' ..x1.. ',0);')


		-- draw bubble annotations
		for r=1, #data do
			if r ~= hlrow then
			local year   = data[r][1]
			local month  = data[r][2]
			local height = data[r][3]
			local text   = data[r][4]
			if month == '' then
				month = 1
			end
			if height == '' then
				height = 0
			end

			local dy  = height * 0.8 + 0.15
			local x = dateX(year,month)
			
			tex.sprint('\\draw [annotation]')
			tex.sprint('(' ..x.. ',0) -- ++(0,' ..dy.. ') -- node [bubble] {\\footnotesize ' ..text.. '} +(A);')
			end
		end


		-- draw highlighted bubble annotation
		local hlx = -1
		if ( hlrow > 0 ) then
			local year   = data[hlrow][1]
			local month  = data[hlrow][2]
			local height = data[hlrow][3]
			local text   = data[hlrow][4]
			if month == '' then
				month = 1
			end
			if height == '' then
				height = 0
			end
			
			hlx = dateX(year,month)
			local dy  = height * 0.8 + 0.15

			-- tex.sprint('\\draw [hlannotation]')
			-- tex.sprint('(' ..hlx.. ',0) -- ++(0,' ..dy.. ') -- node [hlbubble] {\\footnotesize ' ..text.. '} +(A);')
			tex.sprint('\\node [titlebubble] at (10,1) (title) {\\LARGE ' ..text.. '};')
			tex.sprint('\\draw [hlannotation]')
			tex.sprint('(' ..hlx.. ',0) -- ++(0,' ..dy.. ') -- (title);')
		end


		-- draw timeline progress
		tex.sprint('\\draw [->, very thin] (' ..x0.. ',0) -- (' ..x1.. ',0);')
		if ( hlrow > 0 ) then
			tex.sprint('\\draw [hlannotation] (' ..x0.. ',0) -- (' ..hlx.. ',0);')
		end


		-- draw years and ticks
		for year=years[1], years[#years]-1 do
			local x = dateX(year,1)
			local astyle = 'black'
			if x <= hlx then
				astyle = 'hlannotation'
			end
			if year % 5 == 0 then
				tex.sprint('\\draw [' .. astyle .. ', very thin, fill] (' ..x.. ', 0) circle (0.05);')
				tex.sprint('\\node at (' ..x.. ',0) [rotate=45, scale=0.6, anchor=north east] {\\footnotesize ' ..year.. '};')
			else
				tex.sprint('\\draw [' .. astyle .. ', thin, fill] (' ..x.. ', 0) -- +(0,-0.05);')
			end
		end


\end{luacode}
\end{tikzpicture}
\end{document}
